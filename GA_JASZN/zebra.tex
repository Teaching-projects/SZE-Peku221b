\documentclass[12ppt,a4paper,oneside]{report}

\usepackage{makrok}

\author{Szili Dániel, Schöffer Fruzsina, Tóth Sándor Balázs, Varga Máté\\\ \\Témavezető: Dr. Hegyháti Máté}
\title{Zebra típusú logikai rejtvények megoldása evolúciós algoritmussal}

\begin{document}

\maketitle

\tableofcontents

\chapter{Bevezetés} % unassigned
    \todo{Absztrakt bovebben, szoveges tartalomjegyzek}

\chapter{Zebra rejtvények} % unassigned
    \todo{Egy bevezeto mondat, + hogy melyik alfejezetben mi lesz}
    
    \section{Történet és szerkezet} % unassigned
        \todo{Miert zebra, mikbol all a rejtveny, Einstein peldajabol reszlet akar}

    \section{Megoldhatóság, egyértelműség} % unassigned
        \todo{Pici peldakon bemutatni, hogy ha rosszak a szabalyok, akkor lehet nincs megoldas, vagy ha keves a szabaly, akkor lehet tobb megoldas is van. Egy nagyon apro (3 szek, 2 tulajdonsag mondjuk) pelda kitalalasa es megoldasa par lepesben. }

    \section{Megoldó módszerek} % unassigned
        \todo{Irodalomban talalhato modszerek, 2 mondat roluk, hivatkozasok}

\chapter{Evolúciós algoritmusok} % unassigned
    \todo{Tortenetuk, hivatkozasok}
    \todo{Altalanos felepitesuk}
    \todo{Akar par szo arrol, mi mindenre alkalmaztak oket, hivatkozasok}

\chapter{Evolúciós algoritmus Zebra rejtvények megfejtésére} % unassigned
    \todo{Par mondat az alapveto elgondolasrol, hogy melyik fejezetben mirol lesz szo}

    \section{Kódszerkezet} % unassigned
        \todo{Hogy van szervezve a kod, milyen fuggvenyek vannak,azok mikert fognak felelni roviden.}

    \section{Egyedreprezentáció és segédfüggvények} % unassigned
        \todo{Hogy reprezentaljuk az egyedet}
        \todo{egyedkiir, sorbarendez, ...}

    \section{Evolúciós mechanizmusok} % unassigned
        \todo{esetleg par felvezeto szo, a kapcsolodo makrok megemlitese (popmeret, megtart)}
        
        \subsection{Random új egyed generálás} % unassigned
            \todo{Milyen volt az elso valtozat, hogy lett fejlesztve}
            
        \subsection{Mutálás} % unassigned
            \todo{Ugyanez. Milyen valtozatok voltak, vannak, részletesen bemutatva}
            
        \subsection{Keresztezés} % unassigned
            \todo{Ugyanez. Milyen valtozatok voltak, vannak, részletesen bemutatva}

        \subsection{Megold függvény} % unassigned
            \todo{Ugyanez. Milyen valtozatok voltak, vannak, részletesen bemutatva}

    \section{Egyedek kiértékelése} % unassigned
        \todo{Itt is szepen be lehet mutatni, hogy hogyan fejlodott, meg meg lehet mutatni mind az ot-hat tipusra egy peldat}

\chapter{Tesztek} % unassigned
    \todo{Futtatasi eredmenyek, megoldasok megmutatasa, stb.}

\chapter{Kód automatikus generálása} % unassigned
    \todo{Miert akarjuk}
    \todo{Hogy csinaltuk}
    \todo{pelda}

\chapter{Összefoglalás} % unassigned
    \todo{Mit csinaltunk roviden}

\chapter*{Hivatkozások}
\addcontentsline{toc}{chapter}{Hivatkozások}

\appendix

\chapter{Mintafejezet}

\end{document}
